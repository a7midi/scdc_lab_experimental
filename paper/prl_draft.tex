\documentclass[aps,prl,reprint,onecolumn]{revtex4-2}

\usepackage{graphicx}
\usepackage{amsmath}
\usepackage{hyperref}

\begin{document}

\title{Transport, Localization, and Waveguiding in Schedule-Consistent Causal Graphs}

\author{A. Alayyar}
\affiliation{Independent Research}

\date{\today}

\begin{abstract}
We study discrete dynamics on random layered directed acyclic graphs (DAGs) subject to a
\emph{schedule-consistency} constraint: local ``diamond'' subgraphs must evolve to the same state under all
\emph{admissible} update orders. Enforcing this constraint via a Schedule-Consistent Diamond Closure (SCDC)
produces a disordered causal medium with robust, order-independent evolution. Scanning the forward-edge
density reveals three transport regimes for injected excitations—extinction, localization, and runaway
growth—characterized by the horizon-limited light-cone volume. In a separate ``radiation'' experiment, dense
local motifs act as waveguides that amplify detector flux by orders of magnitude without changing causal
reachability. These results show that conduction channels and localized quasi-particle-like pockets can
emerge from geometric constraints in a random causal substrate, without hand-engineering propagating
patterns.
\end{abstract}

\maketitle

\paragraph{Motivation.}
Discrete causal structures have been explored as candidates for microscopic spacetime
\cite{bombelli1987}. A recurring difficulty is ensuring that dynamics is well-defined despite local freedom
in update order: in a DAG, many vertices are spacelike-separated and can be updated in either order.
Physical evolution should not depend on arbitrary scheduling choices. In rewriting theory, this motivates
\emph{confluence}: different reduction orders lead to the same normal form. Here we impose an analogous
principle on causal graphs by enforcing local order-independence on schedule-free diamonds, and then ask:
\emph{What kinds of transport and emergent excitations arise in the resulting disordered medium?}

\paragraph{Model.}
We generate a layered DAG with $L$ layers and $N$ nodes total. Edges point forward in layer index.
A forward-edge probability $p_f$ controls the density of local causal links; optional skip edges
($p_{2},p_{3}$) connect across two or three layers.
Binary activity evolves by a local rule (threshold or XOR) under synchronous updates.

\emph{Schedule-consistency.}
A diamond is a pattern $x\!\to\! y$, $x\!\to\! z$, $y\!\to\! w$, $z\!\to\! w$. If $y$ and $z$ are causally independent
(neither reaches the other), then both update orders $(y \rightarrow z)$ and $(z \rightarrow y)$ are admissible.
SCDC enforces that these admissible orders yield identical outcomes at $w$.
(We explicitly exclude ``diamonds'' where $y$ reaches $z$ or $z$ reaches $y$, where one of the orders is anti-causal.)
The resulting closure produces dynamics that is robust to update-order freedom, while preserving causal structure.

\paragraph{H0: Transport regimes vs density.}
To quantify transport, we inject activity at $n_s=8$ random sources and evolve for $t_{\max}=8$.
For each source we record the \emph{horizon-limited light-cone volume}
\begin{equation}
V_h \equiv |A(t_{\max})| ,
\end{equation}
the number of active nodes at the final tick within the future cone.

We classify each source into one of three coarse regimes:
\begin{enumerate}
\item \textbf{dead}: $V_h < 5$ (extinction),
\item \textbf{localized}: $5 \le V_h \le 20$ (confined propagation),
\item \textbf{shockwave}: $V_h > 20$ (runaway growth).
\end{enumerate}
Figure~\ref{fig:regimes} shows the fraction of sources in each regime as a function of $p_f$
(pooled across 5 seeds; $N=40$ sources per density point in our default scan).
The response is non-monotonic: intermediate densities produce the largest fraction of shockwave events,
while both low and high densities favor extinction or confinement.
This resembles transport in disordered media where conduction depends on the availability of connected
channels \cite{stauffer1994,newman2018}.

\begin{figure}
\includegraphics[width=\linewidth]{../results/fig_transport_regimes.png}
\caption{Transport regimes vs forward-edge density $p_f$ using the horizon-limited volume $V_h$.
Fractions are computed over injected sources (8 per run) pooled across 5 seeds per density.
A non-monotonic response is observed: runaway ``shockwave'' events are most frequent near intermediate $p_f$,
while extinction dominates at low and high $p_f$.}
\label{fig:regimes}
\end{figure}

\paragraph{Critical spectrum: rare large cones at $p_f=0.12$.}
To compare two densities within the scan, Fig.~\ref{fig:spectrum} plots the distribution of $V_h$ at
$p_f=0.12$ and $p_f=0.14$ (40 sources each). While both densities exhibit many small outcomes ($V_h \sim 1$--4),
$p_f=0.12$ supports rare large events ($V_h \gtrsim 20$) that are absent at $p_f=0.14$ for the same thresholds.
This indicates that geometric constraints can suppress a continuum of intermediate excitations and instead
favor either rapid decay or occasional extended bursts—an effect reminiscent of disorder-induced spectral
structure \cite{anderson1958}.

\begin{figure}
\includegraphics[width=\linewidth]{../results/fig_volume_spectrum_pf012_pf014.png}
\caption{Horizon-limited volume spectrum at $p_f=0.12$ vs $p_f=0.14$ (log scale).
Each point is one injected source ($N=40$ per density). The $p_f=0.12$ case supports rare large light-cones
above the shockwave threshold ($V_h > 20$), while $p_f=0.14$ remains confined to smaller outcomes for the
same protocol.}
\label{fig:spectrum}
\end{figure}

\paragraph{H1: Dense motifs as waveguides (radiation experiment).}
We next test whether localized dense subgraphs can act as \emph{conduction filaments} that guide propagation.
In a separate layered-DAG setup, we place a small set of ``knot'' nodes near the source layer and vary only
their internal wiring density. We then excite a fixed initial state and record the detector-layer flux
$\Phi(t)$ (the number of active detector nodes) as a function of time.

Figure~\ref{fig:h1} compares a dense knot ($p_{\mathrm{knot}}=0.9$) to the same knot nodes with no internal
knot wiring ($p_{\mathrm{knot}}=0.0$). The detector is reached at the same first-arrival time in both cases,
consistent with identical topological reachability. However, the dense knot amplifies the \emph{integrated}
detector flux by orders of magnitude (in our representative run, $\sum_t \Phi(t)$ increases by $\sim 10^2$),
indicating that local motif density can strongly focus/guide activity without altering causal accessibility.
This supports a ``filament'' picture in which the geometry creates effective highways for energy transport.

\begin{figure}
\includegraphics[width=\linewidth]{../results/fig_h1_detector_flux.png}
\caption{H1 radiation experiment: detector flux $\Phi(t)$ with and without dense internal wiring among the same
knot nodes. Both cases reach the detector at the same first-arrival tick, but the dense knot produces orders-of-magnitude
higher detector flux, consistent with waveguiding/focusing by local motifs.}
\label{fig:h1}
\end{figure}

\paragraph{Discussion.}
A common approach to propagating patterns in discrete dynamics is to search for rule-engineered ``gliders''
(as in cellular automata). Our results emphasize a complementary mechanism: \emph{geometry-first transport}.
In SCDC-enforced causal DAGs, propagation can become confined to sparse high-conductance subgraphs
(``filaments''), and localized moving patterns arise naturally when activity enters such channels.
The transport scan (H0) maps where the medium is insulating, localizing, or unstable, while the radiation
experiment (H1) shows that dense motifs can function as waveguides that amplify signal transmission.
Together, these experiments suggest that quasi-particle-like pockets and guided transport can emerge
from purely combinatorial constraints in random causal substrates.

\paragraph{Reproducibility.}
All figures and CSVs in this letter are generated by the scripts included with this submission.
Key parameters (graph size, layer count, densities, thresholds, seeds) are explicitly specified in the
repository README, along with one-command reproduction of each figure.

\bibliography{refs}

\end{document}
